% Options for packages loaded elsewhere
\PassOptionsToPackage{unicode}{hyperref}
\PassOptionsToPackage{hyphens}{url}
%
\documentclass[
]{article}
\title{BIOS 611 Project Report}
\usepackage{etoolbox}
\makeatletter
\providecommand{\subtitle}[1]{% add subtitle to \maketitle
  \apptocmd{\@title}{\par {\large #1 \par}}{}{}
}
\makeatother
\subtitle{Global Indicator Data Analysis}
\author{Brooke Felsheim}
\date{}

\usepackage{amsmath,amssymb}
\usepackage{lmodern}
\usepackage{iftex}
\ifPDFTeX
  \usepackage[T1]{fontenc}
  \usepackage[utf8]{inputenc}
  \usepackage{textcomp} % provide euro and other symbols
\else % if luatex or xetex
  \usepackage{unicode-math}
  \defaultfontfeatures{Scale=MatchLowercase}
  \defaultfontfeatures[\rmfamily]{Ligatures=TeX,Scale=1}
\fi
% Use upquote if available, for straight quotes in verbatim environments
\IfFileExists{upquote.sty}{\usepackage{upquote}}{}
\IfFileExists{microtype.sty}{% use microtype if available
  \usepackage[]{microtype}
  \UseMicrotypeSet[protrusion]{basicmath} % disable protrusion for tt fonts
}{}
\makeatletter
\@ifundefined{KOMAClassName}{% if non-KOMA class
  \IfFileExists{parskip.sty}{%
    \usepackage{parskip}
  }{% else
    \setlength{\parindent}{0pt}
    \setlength{\parskip}{6pt plus 2pt minus 1pt}}
}{% if KOMA class
  \KOMAoptions{parskip=half}}
\makeatother
\usepackage{xcolor}
\IfFileExists{xurl.sty}{\usepackage{xurl}}{} % add URL line breaks if available
\IfFileExists{bookmark.sty}{\usepackage{bookmark}}{\usepackage{hyperref}}
\hypersetup{
  pdftitle={BIOS 611 Project Report},
  pdfauthor={Brooke Felsheim},
  hidelinks,
  pdfcreator={LaTeX via pandoc}}
\urlstyle{same} % disable monospaced font for URLs
\usepackage[margin=1in]{geometry}
\usepackage{color}
\usepackage{fancyvrb}
\newcommand{\VerbBar}{|}
\newcommand{\VERB}{\Verb[commandchars=\\\{\}]}
\DefineVerbatimEnvironment{Highlighting}{Verbatim}{commandchars=\\\{\}}
% Add ',fontsize=\small' for more characters per line
\usepackage{framed}
\definecolor{shadecolor}{RGB}{248,248,248}
\newenvironment{Shaded}{\begin{snugshade}}{\end{snugshade}}
\newcommand{\AlertTok}[1]{\textcolor[rgb]{0.94,0.16,0.16}{#1}}
\newcommand{\AnnotationTok}[1]{\textcolor[rgb]{0.56,0.35,0.01}{\textbf{\textit{#1}}}}
\newcommand{\AttributeTok}[1]{\textcolor[rgb]{0.77,0.63,0.00}{#1}}
\newcommand{\BaseNTok}[1]{\textcolor[rgb]{0.00,0.00,0.81}{#1}}
\newcommand{\BuiltInTok}[1]{#1}
\newcommand{\CharTok}[1]{\textcolor[rgb]{0.31,0.60,0.02}{#1}}
\newcommand{\CommentTok}[1]{\textcolor[rgb]{0.56,0.35,0.01}{\textit{#1}}}
\newcommand{\CommentVarTok}[1]{\textcolor[rgb]{0.56,0.35,0.01}{\textbf{\textit{#1}}}}
\newcommand{\ConstantTok}[1]{\textcolor[rgb]{0.00,0.00,0.00}{#1}}
\newcommand{\ControlFlowTok}[1]{\textcolor[rgb]{0.13,0.29,0.53}{\textbf{#1}}}
\newcommand{\DataTypeTok}[1]{\textcolor[rgb]{0.13,0.29,0.53}{#1}}
\newcommand{\DecValTok}[1]{\textcolor[rgb]{0.00,0.00,0.81}{#1}}
\newcommand{\DocumentationTok}[1]{\textcolor[rgb]{0.56,0.35,0.01}{\textbf{\textit{#1}}}}
\newcommand{\ErrorTok}[1]{\textcolor[rgb]{0.64,0.00,0.00}{\textbf{#1}}}
\newcommand{\ExtensionTok}[1]{#1}
\newcommand{\FloatTok}[1]{\textcolor[rgb]{0.00,0.00,0.81}{#1}}
\newcommand{\FunctionTok}[1]{\textcolor[rgb]{0.00,0.00,0.00}{#1}}
\newcommand{\ImportTok}[1]{#1}
\newcommand{\InformationTok}[1]{\textcolor[rgb]{0.56,0.35,0.01}{\textbf{\textit{#1}}}}
\newcommand{\KeywordTok}[1]{\textcolor[rgb]{0.13,0.29,0.53}{\textbf{#1}}}
\newcommand{\NormalTok}[1]{#1}
\newcommand{\OperatorTok}[1]{\textcolor[rgb]{0.81,0.36,0.00}{\textbf{#1}}}
\newcommand{\OtherTok}[1]{\textcolor[rgb]{0.56,0.35,0.01}{#1}}
\newcommand{\PreprocessorTok}[1]{\textcolor[rgb]{0.56,0.35,0.01}{\textit{#1}}}
\newcommand{\RegionMarkerTok}[1]{#1}
\newcommand{\SpecialCharTok}[1]{\textcolor[rgb]{0.00,0.00,0.00}{#1}}
\newcommand{\SpecialStringTok}[1]{\textcolor[rgb]{0.31,0.60,0.02}{#1}}
\newcommand{\StringTok}[1]{\textcolor[rgb]{0.31,0.60,0.02}{#1}}
\newcommand{\VariableTok}[1]{\textcolor[rgb]{0.00,0.00,0.00}{#1}}
\newcommand{\VerbatimStringTok}[1]{\textcolor[rgb]{0.31,0.60,0.02}{#1}}
\newcommand{\WarningTok}[1]{\textcolor[rgb]{0.56,0.35,0.01}{\textbf{\textit{#1}}}}
\usepackage{graphicx}
\makeatletter
\def\maxwidth{\ifdim\Gin@nat@width>\linewidth\linewidth\else\Gin@nat@width\fi}
\def\maxheight{\ifdim\Gin@nat@height>\textheight\textheight\else\Gin@nat@height\fi}
\makeatother
% Scale images if necessary, so that they will not overflow the page
% margins by default, and it is still possible to overwrite the defaults
% using explicit options in \includegraphics[width, height, ...]{}
\setkeys{Gin}{width=\maxwidth,height=\maxheight,keepaspectratio}
% Set default figure placement to htbp
\makeatletter
\def\fps@figure{htbp}
\makeatother
\setlength{\emergencystretch}{3em} % prevent overfull lines
\providecommand{\tightlist}{%
  \setlength{\itemsep}{0pt}\setlength{\parskip}{0pt}}
\setcounter{secnumdepth}{-\maxdimen} % remove section numbering
\usepackage[labelfont={bf}]{caption}
\usepackage[unicode=true, breaklinks=true]{hyperref}
\ifLuaTeX
  \usepackage{selnolig}  % disable illegal ligatures
\fi

\begin{document}
\maketitle

\hypertarget{introduction}{%
\section{1. Introduction}\label{introduction}}

The future existence of humankind is dependent on our ability to live
sustainably. As human populations rise along with greenhouse gas
emissions, deforestation rates, and generation of waste, we will
continue to deplete natural resources, disrupt ecosystems, and increase
global temperatures, leading to an unsustainable future. Because of
this, it is critical to study environmental indicators to assess the
current state and trajectory of the environment.

For my BIOS 611 project, I chose to analyze global environmental
indicator data along with global economic and happiness indicator data.
My goal was to assess recent environmental trends of countries around
the world and to see how these trends might correspond with the state of
the economy and measured levels of happiness within the countries.

\hypertarget{source-data-description}{%
\section{2. Source data description}\label{source-data-description}}

There were three types of source data sets used for this analysis:
environmental indicator data, economic indicator data, and happiness
indicator data. Each data type contains quantitative indicator measures
by country and year.

\hypertarget{environmental-indicator-data}{%
\subsection{Environmental indicator
data}\label{environmental-indicator-data}}

The environmental indicator source data come from the United Nations
Statistics Division (UNSD) / United Nations Environment Programme (UNEP)
Questionairre on Environment Statistics. The data were downloaded via
Kaggle
\href{https://www.kaggle.com/ruchi798/global-environmental-indicators}{here}
(last updated June 5, 2021). Multiple types of environmental indicator
data were used in this analysis and fall under the categories of air and
climate, biodiversity, energy, forest, inland water resources, land and
agriculture, natural disasters, and waste. Environmental indicator data
are available within the year range 1990-2020.

\hypertarget{economic-indicator-data}{%
\subsection{Economic indicator data}\label{economic-indicator-data}}

The economic indicator source data come from the UNSD Human Development
Report and were downloaded via Kaggle
\href{https://www.kaggle.com/frankmollard/income-by-country}{here} (last
updated August 11, 2020). The primary measure of economic activity used
for this analysis was gross domestic product (GDP) by country. Economic
indicator data are available within the year range 1990-2018.

\hypertarget{happiness-indicator-data}{%
\subsection{Happiness indicator data}\label{happiness-indicator-data}}

The happiness indicator data come from the World Happiness Report
published by the Sustainable Development Solutions Network. The data
were downloaded via Kaggle
\href{https://www.kaggle.com/unsdsn/world-happiness}{here} (last updated
November 26, 2019). Each country is given a ``happiness score'' (0 to
10) that is based on life evaluation survey responses. Happiness
indicator data are available within the year range 2015-2019.

\hypertarget{results}{%
\section{3. Results}\label{results}}

\hypertarget{exploration-of-indicator-trends-within-countries}{%
\subsection{Exploration of indicator trends within
countries}\label{exploration-of-indicator-trends-within-countries}}

The first goal of my analysis was to explore trends of indicator data
within individual countries. To achieve this goal, I created an
interactive R shiny app that plots many different types indicator data
over time for 190 different countries. The country of interest can first
be selected via a drop-down menu in the app. For the selected country,
thirteen different types of plots are generated:

\begin{itemize}
\tightlist
\item
  Environmental indicator plots

  \begin{itemize}
  \tightlist
  \item
    Greenhouse gas emissions by type over time
  \item
    Greenhouse gas emissions by sector
  \item
    Energy supply per capita over time
  \item
    Renewable energy production percentage over time
  \item
    Forest area over time
  \item
    Precipitation over time
  \item
    Natural disaster occurrences over time
  \item
    Natural disaster deaths over time
  \item
    Hazardous waste by type over time
  \item
    Municipal waste recycled over time
  \end{itemize}
\item
  Economic indicator plots

  \begin{itemize}
  \tightlist
  \item
    Gross domestic product per capita over time
  \item
    Gross national income by gender over time
  \end{itemize}
\item
  Happiness indicator plots

  \begin{itemize}
  \tightlist
  \item
    Happiness score over time
  \end{itemize}
\end{itemize}

The plots displayed in the shiny app can give insight into the level and
ways that a country may be negatively affecting the environment, the
status of a country's economy, and the estimated happiness level of a
country's citizens over time.

As an example, we can look at all of the indicator plots generated for
Sweden in the shiny app. From this data, we can see that Sweden's
greenhouse gas emissions have been decreasing over time, and that most
of these greenhouse gas emissions come from energy use. Correspondingly,
energy supply per capita has been decreasing over time and the total
percentage renewable energy production increasing over time. The total
forest area by year in Sweden increased from 1990-2000, but decreased
from 2000-2020. While the total precipitation fluctuates year by year in
Sweden, the indicator plot shows a general trend of increased
precipitation since 1990. Additionally, Sweden has had very few recent
natural disasters, treats/disposes of approximately half of its
hazardous waste, and has been increasing the percentage of municipal
waste it recycles. We can also see that Sweden's GDP has been steadily
rising over time, and while the national income has been rising as well,
it remains higher for men than women. Furthermore, Sweden's happiness
score has only fluctuated by less than 0.1 out of 10 from 2015-19.

We can also notice some other important trends from the indicator plots
generated by the shiny app.

\begin{figure}
\centering
\includegraphics{"figures/top_10_energy_countries.png"}
\caption{Energy trends for top ten energy-consuming countries.
\textbf{(A)} Energy supply by country over time. \textbf{(B)} Percent of
total energy used that is renewable over time.}
\end{figure}

\begin{figure}
\centering
\includegraphics{"figures/top_10_energy_per_capita_countries.png"}
\caption{Energy trends for top ten energy-consuming countries per
capita}
\end{figure}

\hypertarget{exploration-of-trends-between-indicators}{%
\subsection{Exploration of trends between
indicators}\label{exploration-of-trends-between-indicators}}

\begin{figure}
\centering
\includegraphics{"figures/paired_indicators.png"}
\caption{Paired indicators}
\end{figure}

\begin{Shaded}
\begin{Highlighting}[]
\FunctionTok{readRDS}\NormalTok{(}\StringTok{"outputs/environmental\_indicator\_pc\_summary.rds"}\NormalTok{)}
\end{Highlighting}
\end{Shaded}

\begin{verbatim}
## Importance of components:
##                           PC1    PC2    PC3    PC4     PC5    PC6
## Standard deviation     1.4154 1.2902 0.9436 0.8999 0.66219 0.4395
## Proportion of Variance 0.3339 0.2774 0.1484 0.1350 0.07308 0.0322
## Cumulative Proportion  0.3339 0.6113 0.7597 0.8947 0.96780 1.0000
\end{verbatim}

\begin{figure}
\centering
\includegraphics{"figures/environmental_indicator_pc_plot"}
\caption{Environmental indicator PCA}
\end{figure}

\begin{figure}
\centering
\includegraphics{"figures/region_boxplots.png"}
\caption{Region boxplots}
\end{figure}

\hypertarget{prediction-of-happiness-level-from-environmental-indicator-data}{%
\subsection{Prediction of happiness level from environmental indicator
data}\label{prediction-of-happiness-level-from-environmental-indicator-data}}

\begin{Shaded}
\begin{Highlighting}[]
\FunctionTok{readRDS}\NormalTok{(}\StringTok{"outputs/happiness\_elasticnet\_model.rds"}\NormalTok{)}
\end{Highlighting}
\end{Shaded}

\begin{verbatim}
## glmnet 
## 
## 86 samples
##  6 predictor
##  2 classes: 'Low', 'High' 
## 
## No pre-processing
## Resampling: Cross-Validated (10 fold) 
## Summary of sample sizes: 77, 77, 78, 77, 78, 78, ... 
## Resampling results across tuning parameters:
## 
##   alpha  lambda        Accuracy   Kappa    
##   0.10   0.0005588992  0.7805556  0.5610976
##   0.10   0.0055889918  0.7594444  0.5202439
##   0.10   0.0558899181  0.7594444  0.5202439
##   0.55   0.0005588992  0.7805556  0.5610976
##   0.55   0.0055889918  0.7594444  0.5202439
##   0.55   0.0558899181  0.7783333  0.5603833
##   1.00   0.0005588992  0.7805556  0.5610976
##   1.00   0.0055889918  0.7694444  0.5402439
##   1.00   0.0558899181  0.7758333  0.5553833
## 
## Accuracy was used to select the optimal model using the largest value.
## The final values used for the model were alpha = 0.1 and lambda = 0.0005588992.
\end{verbatim}

\begin{Shaded}
\begin{Highlighting}[]
\FunctionTok{readRDS}\NormalTok{(}\StringTok{"outputs/happiness\_elasticnet\_coefficients.rds"}\NormalTok{)}
\end{Highlighting}
\end{Shaded}

\begin{verbatim}
## 7 x 1 sparse Matrix of class "dgCMatrix"
##                                    s1
## (Intercept)               -4.02755295
## GHG_per_capita_emissions   0.29352022
## Energy_per_capita          0.02136925
## Renewable_energy_percent   0.01427311
## Agricultural_area_percent  0.01430855
## Forest_area_percent        0.01609668
## Protected_area_percent    -0.03236558
\end{verbatim}

\begin{figure}
\centering
\includegraphics{"figures/happiness_elasticnet_figures.png"}
\caption{Happiness predictor}
\end{figure}

\hypertarget{prediction-of-gdp-level-from-environmental-indicator-data}{%
\subsection{Prediction of GDP level from environmental indicator
data}\label{prediction-of-gdp-level-from-environmental-indicator-data}}

\begin{Shaded}
\begin{Highlighting}[]
\FunctionTok{readRDS}\NormalTok{(}\StringTok{"outputs/GDP\_elasticnet\_model.rds"}\NormalTok{)}
\end{Highlighting}
\end{Shaded}

\begin{verbatim}
## glmnet 
## 
## 86 samples
##  6 predictor
##  2 classes: 'Low', 'High' 
## 
## No pre-processing
## Resampling: Cross-Validated (10 fold) 
## Summary of sample sizes: 77, 77, 78, 77, 78, 78, ... 
## Resampling results across tuning parameters:
## 
##   alpha  lambda        Accuracy   Kappa    
##   0.10   0.0006950761  0.9305556  0.8599719
##   0.10   0.0069507607  0.8844444  0.7680206
##   0.10   0.0695076065  0.9094444  0.8180206
##   0.55   0.0006950761  0.9305556  0.8599719
##   0.55   0.0069507607  0.8944444  0.7880206
##   0.55   0.0695076065  0.9094444  0.8180206
##   1.00   0.0006950761  0.9208333  0.8399719
##   1.00   0.0069507607  0.9319444  0.8630206
##   1.00   0.0695076065  0.8969444  0.7930206
## 
## Accuracy was used to select the optimal model using the largest value.
## The final values used for the model were alpha = 1 and lambda = 0.006950761.
\end{verbatim}

\begin{Shaded}
\begin{Highlighting}[]
\FunctionTok{readRDS}\NormalTok{(}\StringTok{"outputs/GDP\_elasticnet\_coefficients.rds"}\NormalTok{)}
\end{Highlighting}
\end{Shaded}

\begin{verbatim}
## 7 x 1 sparse Matrix of class "dgCMatrix"
##                                     s1
## (Intercept)               -4.746593788
## GHG_per_capita_emissions   0.240806557
## Energy_per_capita          0.066914612
## Renewable_energy_percent  -0.012684575
## Agricultural_area_percent -0.005305846
## Forest_area_percent        0.011387377
## Protected_area_percent     0.022893270
\end{verbatim}

\begin{figure}
\centering
\includegraphics{"figures/GDP_elasticnet_figures.png"}
\caption{GDP predictor}
\end{figure}

\hypertarget{conclusions}{%
\section{4. Conclusions}\label{conclusions}}

\hypertarget{further-exploration}{%
\section{5. Further exploration}\label{further-exploration}}

\begin{center}\rule{0.5\linewidth}{0.5pt}\end{center}

\end{document}
